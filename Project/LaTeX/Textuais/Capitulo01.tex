\chapter{Introdução}

    Atualmente vivemos em um cenário global muito instável em relação a diversos pontos que nos impactam diretamente,
    portanto, sempre precisamos estar preparados para momentos de recessão onde muitas vezes quem não tem algum tipo de
    reserva, ou ela e insuficiente para mantar a família pelo período de crise, acaba tendo dificuldades de manter seu
    padrão de vida e conforto, logo, a educação financeira precisa estar introduzida na educação desde seus primórdios.
    Deste modo, tendo dimensão da importância do assunto, foi instaurada, por meio do decreto 7.397/2010, e atualizado
    pelo Decreto nº 10.393/2020 a Estratégia Nacional da Educação Financeira -- ENEF e o Fórum Brasileiro de Educação
    Financeira -- FBEF \cite{decreto_10939}.

    A abundância de campanhas de \textit{marketing} que criam tendências e hábitos de consumo, também pode se tornar um
    sério problema para o consumista, exigindo assim uma considerável quantidade de autocontrole e planejamento para não
    acabar consumindo serviços ou adquirindo bens com os quais não se conseguirão honrar.
    Para \citeshort{Junger_Medeiros_Moura_Barrocal_Amaral_2019}, isso se acentua muito quando olhamos para as campanhas
    com público alvo infantil, que por possuir um conhecimento mais limitado, acaba por ser muito afetado negativamente,
    confundindo os sentidos de necessidade e desejo, aumentando assim as chances de que esta criança venha a se tornar
    um adulto consumista.

    A educação financeira atualmente é um tema muito discutido, uma vez que quanto mais elevado for o nível de
    conhecimento do indivíduo sobre como aplicá-la em seu dia a dia, mais tranquila sua vida tende a ser, pois quando os
    riscos são considerados e um plano de ação é traçado, já estamos preparados para ambos os cenários. De acordo com
    \citeshort{OECD}, a educação financeira é definida de maneira a facilitar o entendimento a cerca de conceitos
    financeiros, riscos e oportunidades, a fim de melhorar as habilidades de análise de caminhos a seguir, para que
    desta maneira tenha um melhor discernimento sobre possíveis oportunidades ou riscos de perda que lhe surgem na vida.

    \section{Problema}

        O sistema tradicional de ensino, com aulas expositivas, não é mais eficiente para atrair a atenção de alunos
        que, na verdade, estão acostumados com conteúdos que interagem com eles. A inovação na educação é, portanto, uma
        área a ser explorada, principalmente para o desenvolvimento de soluções inovadoras que sejam capazes de tornar a
        educação mais interessante para os alunos.

        Ao utilizar jogos digitais para fins educacionais, os alunos têm a oportunidade de aprender de forma mais
        prazerosa e interativa. Isso acontece porque os jogos despertam o interesse dos alunos, fazendo com que eles se
        envolvem mais com o conteúdo apresentado. Além disso, os jogos também ajudam a estimular a criatividade e a
        imaginação dos alunos, o que pode ser muito útil para o desenvolvimento de novas habilidades.  Outro fator que
        pode ser percebido é o aumento da socialização entre os alunos, pois através dos jogos eles são obrigados a se
        relacionar e trabalhar em grupo para chegar ao objetivo final. Porém, apesar dos benefícios, é importante
        ressaltar que a aplicação de jogos digitais na educação infantil deve ser feita de forma controlada, pois os
        jogos necessitam de uma avaliação dos docentes antes da aplicação para entender se ele é adequado a idade do
        aluno \cite{Cruz_Araujo_Andrye_Galvao_Madeira_2022}.

        Desta maneira, percebeu-se que a necessidade da implementação de um jogo digital que lecionasse alguns
        conteúdos sobre educação financeira para crianças do ensino fundamental de forma que elas possam avançar em suas
        vidas de forma mais consciente sobre sua vida financeira.

    \section{Objetivos}

        Esta seção apresenta o objetivo geral e os objetivos específicos do presente trabalho.

        \subsection{Objetivos Gerais}

            Implementar um software que auxilia o ensino, este,  que será no formato de um jogo que busca transmitir
            alguns conhecimentos sobre educação financeira, para permitir ao aluno tomar decisões que o levam para
            múltiplos finais, alguns bons, outros nem tanto, desta forma, trazendo uma consciência sobre consumo e gastos.
            Para medir se houve um real benefício na aplicação do jogo, será aplicado um teste de conhecimentos antes e
            depois da aplicação do jogo.

        \subsection{Objetivos Específicos}

            \begin{enumerate}[noitemsep,nosep,labelindent=\parindent,leftmargin=*,label={\alph*}) ]
                \item Digitalizar os 3 jogos disponíveis no material disponibilizado \cite{Educacao_financeira_nas_escolas}.
                \item Aplicar um teste de conhecimento nos participantes, antes e outro após a aplicação dos jogos.
                \item Baseado nos resultados obtidos, entender se há real eficácia na aplicação dos jogos.
            \end{enumerate}

    \section{Justificativa}

        As práticas lúdicas são atividades que trazem aprendizado através de brincadeiras ou jogos, são essenciais na
        formação do indivíduo, pois fomentam aptidões tais como a criatividade, empatia, interação social. Isso é mais
        efetivo no aprendizado da criança do que utilizar apenas práticas convencionais, pois é sem dúvida mais simples
        manter uma criança atenta durante uma brincadeira do que durante uma explicação convencional, principalmente com
        a disponibilidade de outras opções de entretenimento bem na palma de suas mãos \cite{Santos_Thayna_da_silva_2021}.

        Visando tomar proveito dos benefícios citados anteriormente das atividades lúdicas, será realizada a
        implementação a versão digital do jogo disponibilizado no material do 5º ano do ensino fundamental para ser mais
        lúdico no sentido de permitir ao aluno interagir de maneira mais imersiva em relação ao modelo convencional.
        Atualmente, o referido jogo precisa ser jogado folhando as páginas do livro com muito texto e poucas imagens, o
        que de certa forma diminui o interesse da criança pela atividade.

    \section{Metodologia}

        A fim de atender aos objetivos do presente trabalho, será realizada a digitalização de 3 jogos presentes no
        material disponibilizado para o 5º ano do ensino fundamental pela ENEF. No momento, estes jogos estão
        disponíveis em um formato um pouco menos intuitivo para crianças de 10 anos, onde para que elas joguem, precisam
        folhar um livro com bastante texto e pouca interatividade, reduzindo desta maneira, o nível de concentração e o
        aproveitamento dos ensinamentos repassados.

        A digitalização do jogo consistirá no levantamento dos fluxos de caminhos possíveis de cada jogo, que permitirá
        ao jogador tomar decisões que o levarão por caminhos diferentes e implicarão em consequências no andamento do
        jogo, que será divido em cenários, onde cada cenário representa uma cena da história. Isso tudo para garantir
        que o jogador permanecerá focado durante todo período que passar jogando.

        A implementação do jogo ocorrerá utilizando o \textit{framework} RPGJS, que permite a criação de um RPG ou MMORPG, isso
        possibilita a criação de mapas com cada cenário, provendo uma experiência mais imersiva para o usuário.
        Outro ponto importante na escolha da tecnologia foi a plataforma de destino, sendo a web, que capacita este
        \textit{framework} para disponibilizar o jogo sem a necessidade da instalação e em qualquer dispotivo que tenha
        acesso a internet, o que permite atingir um público muito maior.

        Ao final do desenvolvimento ocorrerá a implementação de um questionário, que será aplicado antes de após a
        aplicação do jogo, isso para avaliar o real aprendizado dos alunos ao jogarem os jogos.
