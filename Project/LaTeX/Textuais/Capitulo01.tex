\chapter{Introdução}

    Atualmente vivemos em um cenário global muito instável em relação a diversos pontos que nos impactam diretamente,
    portanto, sempre precisamos estar preparados para momentos de recessão onde muitas vezes quem não tem algum tipo de
    reserva, ou ela e insuficiente para mantar a família pelo período de crise, acaba tendo dificuldades de manter seu
    padrão de vida e conforto, logo, a educação financeira precisa estar introduzida na educação desde seus primórdios.
    Deste modo, tendo dimensão da importância do assunto, foi instaurada, por meio do decreto 7.397/2010, e atualizado
    pelo Decreto nº 10.393/2020 a Estratégia Nacional da Educação Financeira -- ENEF e o Fórum Brasileiro de Educação
    Financeira -- FBEF.\cite{decreto_10939}

    A abundância de campanhas de \textit{marketing} que criam tendências e hábitos de consumo, também pode se tornar um
    sério problema para o consumista, exigindo assim uma considerável quantidade de autocontrole e planejamento para não
    acabar consumindo serviços ou adquirindo bens com os quais não se conseguirão honrar.
    Para \citeshort{Junger_Medeiros_Moura_Barrocal_Amaral_2019}, isso se acentua muito quando olhamos para as campanhas
    com público alvo infantil, que por possuir um conhecimento mais limitado, acaba por ser muito afetado negativamente,
    confundindo os sentidos de necessidade e desejo, aumentando assim as chances de que esta criança venha a se tornar
    um adulto consumista.

    A educação financeira atualmente é um tema muito discutido, uma vez que quanto mais elevado for o nível de
    conhecimento do indivíduo sobre como aplicá-la em seu dia a dia, mais tranquila sua vida tende a ser, pois quando os
    riscos são considerados e um plano de ação é traçado, já estamos preparados para ambos os cenários. De acordo com
    \citeshort{OECD}, a educação financeira é definida de maneira a facilitar o entendimento a cerca de conceitos
    financeiros, riscos e oportunidades, a fim de melhorar as habilidades de análise de caminhos a seguir, para que
    desta maneira tenha um melhor discernimento sobre possíveis oportunidades ou riscos de perda que lhe surgem na vida.

    \section{Problema}

    Atualmente

    \section{Objetivos}

        \subsection{Objetivos Gerais}

        \subsection{Objetivos Específicos}

    \section{Justificativa}

        As práticas lúdicas são atividades que trazem aprendizado através de brincadeiras ou jogos, são essenciais na
        formação do indivíduo, pois fomentam aptidões tais como a criatividade, empatia, interação social. Isso é mais
        efetivo no aprendizado da criança do que utilizar apenas práticas convencionais, pois é sem dúvida mais simples
        manter uma criança atenta durante uma brincadeira do que durante uma explicação convencional, principalmente com
        a disponibilidade de outras opções de entretenimento bem na palma de suas mãos. \cite{Santos_Thayna_da_silva_2021}

    \section{Métodologia}