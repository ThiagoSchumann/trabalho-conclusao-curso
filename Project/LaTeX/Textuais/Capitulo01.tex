\chapter{Introdução}

    A educação financeira é um importante eixo da formação do ser humano, uma vez que quanto mais elevado for o nível de
    conhecimento do indivíduo sobre como aplicá-la em seu dia a dia, mais tranquila sua vida tende a ser. O conhecimento
    sobre gerenciamento de finanças permite ao indivíduo enfrentar um cenário econômico por vezes instável, preparando-o
    para crises e momentos de recessão. De acordo com a\citeshort{OECD}, a educação financeira objetiva facilitar o
    entendimento a cerca de conceitos financeiros, riscos e oportunidades, a fim de melhorar as habilidades de análise
    de caminhos a seguir. Dessa forma, o indivíduo terá melhor discernimento sobre possíveis oportunidades ou riscos de
    perda que lhe surgirem na vida.

    Crianças e adolescentes são mais suscetíveis ao incentivo ao consumismo. Campanhas de \textit{marketing} criam
    tendências e hábitos de consumo, exigindo autocontrole e planejamento para não acabar adquirindo bens ou serviços
    desnecessários, desestruturando a saúde financeira do consumista.
    Para \citeshort{Junger_Medeiros_Moura_Barrocal_Amaral_2019}, isso se acentua muito nas campanhas voltadas ao público
    infantil, que por possuir conhecimento mais limitado, é afetado negativamente e confunde a necessidade com o desejo.
    Isso aumenta as chances da criança se tornar um adulto consumista.

    Por isso, muitos defendem que a educação financeira deve ser desenvolvida desde os primeiros anos da escola, razão
    pela qual o Governo Federal instaurou a Estratégia Nacional da Educação Financeira (ENEF) e o Fórum Brasileiro de
    Educação Financeira (FBEF), através dos Decretos 7.397/2010 e 10.393/2020, respectivamente \cite{decreto_10939}.
    Como parte da estratégia foram criados materiais para o ensino da educação financeira desde as séries iniciais do
    Ensino Fundamental, até o fim do ensino médio. Para o presente trabalho, será utilizado o material disponibilizado
    para o 5º ano do Ensino Fundamental\cite{Educacao_financeira_nas_escolas}.

    \section{Problema}

    Apesar do ensino tradicional ter sido mais eficiente para atrair a atenção de alunos, a recente tendência é a de que
    ele seja menos eficiente, principalmente em termos de aprendizagem. Isso porque as aulas expositivas são mais
    sensíveis aos diferentes tipos de conteúdos, o que fecha ou reduz o interesse dos alunos em estudar. A inovação na
    educação é, portanto, uma área a ser explorada, principalmente para o desenvolvimento de soluções inovadoras que
    consigam tornar a educação mais interessante para os alunos.

    Segundo \citeshort{SERASA_Setembro_2022}, no mês de setembro de 2022 haviam aproximadamente 69 milhões de brasileiros
    inadimplentes, este número representa mais de 30\% da população, conforme estimativa apresentada
    pelo\cite{IBGE_populacao}. Isso demonstra a importância da educação financeira no cenário atual do país, a imersão
    das crianças e adolescentes precisa ocorrer, para que no futuro este número de inadimplentes seja reduzido.

    Ao utilizar jogos digitais para fins educacionais, os alunos têm a oportunidade de aprender de forma mais prazerosa
    e interativa. Isso acontece porque os jogos despertam o interesse dos alunos, fazendo com que eles se envolvam mais
    com o conteúdo apresentado. Além disso, os jogos ajudam a estimular a criatividade e a imaginação dos alunos, o que
    pode ser útil para o  desenvolvimento de novas habilidades. Outro fator que pode ser percebido é o aumento da
    socialização entre os alunos, pois através dos jogos eles são incentivados a se relacionar e trabalhar em grupo para
    chegar ao objetivo final. Porém, apesar dos benefícios, é importante ressaltar que a aplicação de jogos digitais na
    educação infantil deve ser feita controladamente, pois os jogos necessitam de uma avaliação dos docentes antes da
    aplicação para entender se ele é adequado à idade do aluno \cite{Cruz_Araujo_Andrye_Galvao_Madeira_2022}.

    Desta maneira, percebeu-se que a necessidade da implementação de um jogo digital que auxilie no aprendizado de alguns
    conteúdos sobre educação financeira para crianças do Ensino Fundamental.

    \section{Objetivos}

        Esta seção apresenta o objetivo geral e os objetivos específicos do presente trabalho.

        \subsection{Objetivo Geral}

            Desenvolver um jogo digital que auxilie a aprendizagem de conceitos da educação financeira.

        \subsection{Objetivos Específicos}

            \begin{enumerate}[noitemsep,nosep,labelindent=\parindent,leftmargin=*,label={\alph*}) ]

                \item Estudar material disponibilizado pela Estratégia Nacional da Educação Financeira (ENEF), para o 5º
                        ano do Ensino Fundamental \cite{Educacao_financeira_nas_escolas}.

                \item Desenvolver um jogo digital com a primeira de três histórias disponíveis no material disponibilizado.

                \item Disponilizar o jogo em turmas do 5º ano do Ensino Fundamental.

                \item Aplicar o teste sugerido no modelo MEEGA+KIDS \cite{MEEGA+KIDS} nas turmas aplicadas.

                \item Avaliar eficiência do jogo em relação aos conceitos apresentados na metodologia MEEGA+KIDS.

            \end{enumerate}

    \section{Justificativa}

    As práticas lúdicas são atividades que trazem aprendizado através de brincadeiras ou jogos. Elas são importantes na
    formação do indivíduo, pois fomentam aptidões tais como a criatividade, empatia e interação social. Isso é mais
    efetivo no aprendizado da criança do que as práticas convencionais, pois é mais simples manter uma criança atenta
    durante uma brincadeira em comparação a uma explicação mais séria sobre conhecimentos similares, principalmente
    considerando a disponibilidade de outras opções de entretenimento disponíveis \cite{Santos_Thayna_da_silva_2021}.

    Visando tomar proveito dos benefícios das atividades lúdicas, será realizada a implementação da versão digital do
    jogo disponibilizado no material do 5º ano do Ensino Fundamental para ser mais lúdico no sentido de permitir ao
    aluno interagir de maneira mais imersiva em relação ao modelo convencional. Atualmente, o referido jogo precisa ser
    jogado folhando as páginas do livro com muito texto e poucas imagens, o que de certa forma diminui o interesse da
    criança pela atividade.

    \section{Metodologia}

        A fim de atender aos objetivos do presente trabalho, será realizada a digitalização de um de três jogos
        presentes no material disponibilizado para o 5º ano do Ensino Fundamental pela ENEF. No momento, esses jogos
        estão disponíveis em um formato de livro-jogo, que o torna menos intuitivo. A digitalização do jogo consistirá
        no levantamento dos cenários e fluxos de caminhos possíveis, permitindo ao jogador tomar decisões que o levarão
        por caminhos diferentes e implicarão em consequências em seu andamento. O jogo será divido em cenários,
        onde cada cenário representa uma pequena parte da história.

        A implementação do jogo ocorrerá utilizando o \textit{framework} RPGJS, que permite a criação de um
        \textit{Role-Playing Game} (RPG), isso viabiliza a criação de mapas com cada cenário, provendo uma experiência
        imersiva para o usuário. Outro ponto importante na escolha da tecnologia foi a plataforma de destino, sendo a
        \textit{web}, que capacita este \textit{framework} para disponibilizar o jogo sem a necessidade da instalação e
        em qualquer dispotivo que tenha acesso a internet, o que permite atingir um público maior.

        A disponibilização do jogo será realizada em algumas escolas de Ensino Fundamental a fim de permitir que os
        alunos joguem o jogo e após isso estejam aptos a responder o questionário de avaliação.

        A avaliação da eficiência do jogo ocorrerá através da metodologia MEEGA+KIDS, que tem como base a metodologia
        MEEGA+. Ela se resume a um modelo de avaliação de jogos voltados ao ensino, de maneira que propõe um
        questionário baseado em conceitos consolidados de mensuração de atributos mínimos de usabilidade, experiência e
        aprendizado.

