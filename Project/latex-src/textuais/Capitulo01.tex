\chapter{Introdução}

A educação financeira representa um eixo fundamental na formação do ser humano. Segundo a \cite{OECD}, as crianças de hoje têm acesso ao dinheiro e começam a usar serviços financeiros digitais desde cedo. Elas estão crescendo em uma paisagem financeira que evolui rapidamente, o que significa mais oportunidades, mas também mais responsabilidade individual nas decisões financeiras do que as gerações anteriores. É importante que as crianças entendam conceitos financeiros chave e comecem a desenvolver atitudes e comportamentos financeiramente astutos desde uma idade precoce.

Neste cenário, crianças e adolescentes são frequentemente expostos a um vasto volume de informações e estímulos, muitos oriundos de intensas campanhas de marketing, que procuram influenciar seus hábitos de consumo desde a tenra idade. Essas campanhas, muitas vezes sofisticadas, não só promovem o consumo, mas também podem distorcer a percepção entre necessidades e desejos, como discutido por \cite{Junger_Medeiros_Moura_Barrocal_Amaral_2019}. Esta exposição precoce aos conceitos de consumo pode predispor os jovens a padrões de consumo impulsivo na maturidade.

A atuação do Governo Federal do Brasil, por meio da implementação da Estratégia Nacional de Educação Financeira (ENEF) e do Fórum Brasileiro de Educação Financeira (FBEF), estabelecidas respectivamente pelos Decretos nº 7.397/2010 e nº 10.393/2020 \cite{decreto_10939}, é vital. Estas políticas visam incorporar a educação financeira ao currículo escolar desde os anos iniciais, proporcionando aos estudantes uma base robusta para o desenvolvimento de competências financeiras.

A implementação da educação financeira nas escolas produz efeitos positivos em toda a sociedade. Além do benefício direto aos estudantes, suas famílias também são impactadas positivamente. Segundo \cite{Frisancho2023Spillover}, isso é particularmente significativo para famílias em situações socioeconômicas desfavoráveis, onde a educação financeira pode ser um vetor de transformação. \cite{Ozkale2023Designing} complementa ao sugerir que a integração de tarefas matemáticas com conceitos financeiros desde cedo estabelece uma base sólida para a literacia financeira contínua.

Atualmente, os jogos educativos digitais têm se destacado como ferramentas eficazes para aumentar o engajamento e a retenção dos alunos. \cite{lee2023} ressalta que a gamificação, que incorpora elementos de jogos em ambientes educacionais, exerce um impacto positivo no engajamento e no aprendizado dos alunos, especialmente em disciplinas como matemática. \cite{dahalan2023} também enfatiza que a gamificação pode aprimorar o desempenho acadêmico e a motivação dos estudantes.

Conforme destacado por \cite{masnan2016financial}, a educação financeira é essencial desde a infância, preparando as crianças para se tornarem consumidores e gestores competentes. A introdução deste ensino nas escolas inicia o processo de capacitação das crianças para enfrentar futuros desafios financeiros, desenvolvendo habilidades, atitudes e conhecimentos essenciais sobre literacia financeira. Estas competências são ainda mais necessárias em vista das mudanças nos padrões de emprego e das elevadas taxas de desemprego em muitos países. Assim, o desenvolvimento de um programa de educação financeira eficaz, eficiente e viável torn


\section{Problema}

O ensino tradicional, uma abordagem comum nas escolas, enfrenta desafios em manter o engajamento e o interesse dos alunos na era atual. Embora ainda tenha seu valor, a eficácia das aulas expositivas tem sido questionada. De acordo \cite{fortes2023aprendizagem}, a busca por métodos educacionais inovadores é cada vez mais relevante para enriquecer o processo de aprendizagem. A variedade de conteúdos e a complexidade das matérias às vezes não são suficientes para manter a atenção dos estudantes, tornando a inovação educacional uma necessidade.

A situação financeira da população brasileira destaca a urgência dessa inovação. Conforme dados recentes do \cite{SERASA_2023}, em janeiro de 2023, o número de brasileiros inadimplentes atingiu um recorde na série histórica, chegando a 70,1 milhões. Isso representa aproximadamente 35\% da população, que alcançou 203 milhões de habitantes, segundo o \cite{IBGE_Censo_2022}. Essas estatísticas são alarmantes e reforçam a necessidade de uma educação financeira sólida desde os primeiros anos de vida. A inadimplência não é apenas um problema econômico, mas também social, e sua prevenção começa com a educação.


Nesse contexto, jogos digitais surgem como uma ferramenta educacional promissora. Para \cite{Cruz_Araujo_Andrye_Galvao_Madeira_2022}, eles têm o potencial de transformar a aprendizagem em uma experiência mais envolvente e interativa. Além de manter os alunos interessados, os jogos estimulam a criatividade, a imaginação e incentivam a colaboração e o trabalho em equipe. No entanto, é essencial que a introdução de jogos no ambiente educacional seja feita de maneira ponderada. Os educadores devem avaliar cuidadosamente a adequação dos jogos à faixa etária e ao conteúdo programático, garantindo que eles complementem, e não substituam, o currículo tradicional.

Portanto, diante dos desafios apresentados, surge a proposta de desenvolver um jogo digital focado auxílio do ensino de conceitos de educação financeira para alunos do Ensino Fundamental. O objetivo é utilizar essa ferramenta como um meio complementar para enriquecer o processo de aprendizagem e preparar os alunos para enfrentar os desafios financeiros da vida adulta.

\section{Objetivos}
Esta seção apresenta o objetivo geral e os objetivos específicos do presente trabalho.

\subsection{Objetivo Geral}
Desenvolver e implementar um jogo educativo digital para otimizar a aprendizagem de conceitos essenciais de educação financeira para estudantes do 5º ano do Ensino Fundamental.

\subsection{Objetivos Específicos}
\begin{enumerate}[noitemsep,nosep,labelindent=\parindent,leftmargin=*,label={\alph*}) ]
	\item Examinar detalhadamente o conteúdo de educação financeira fornecido pela Estratégia Nacional da Educação Financeira (ENEF) para o 5º ano \cite{Educacao_financeira_nas_escolas}.
	\item Desenhar e desenvolver um jogo digital, inspirado na primeira narrativa do material da ENEF.
	\item Aplicar o jogo com crianças correspondentes a faixa etária (9 aos 11 anos).
	\item Aplicar a avaliação do modelo MEEGA+KIDS \cite{GresseVonWangenheim2020} aos alunos que interagiram com o jogo.
	\item Medir o impacto do jogo na compreensão dos alunos sobre conceitos financeiros, utilizando os padrões da metodologia MEEGA+KIDS \cite{GresseVonWangenheim2020}.
\end{enumerate}

\section{Justificativa}

A aprendizagem através de práticas lúdicas, que engloba atividades centradas em brincadeiras e jogos, tem sido reconhecida por sua capacidade intrínseca de engajar e enriquecer a formação do indivíduo \cite{Santos_Thayna_da_silva_2021}. Estas práticas não apenas cultivam habilidades cognitivas, mas também nutrem competências socioemocionais, como criatividade, empatia e capacidade de interação social. A natureza interativa e envolvente das atividades lúdicas é especialmente saliente no contexto educacional infantil, onde a atenção e o engajamento podem ser desafiantes de serem mantidos através de métodos pedagógicos tradicionais.

Dentro deste panorama, a proposta deste trabalho é aprimorar a experiência educacional ao digitalizar um jogo didático destinado ao 5º ano do Ensino Fundamental. A motivação por trás desta iniciativa é dupla: primeiro, adaptar-se à crescente inclinação das crianças por interações digitais e, segundo, enriquecer o material pedagógico tornando-o mais interativo e visualmente atraente. A versão atual do jogo, predominantemente textual e com limitações visuais, corre o risco de não capturar plenamente o interesse dos estudantes. Portanto, através da digitalização e enriquecimento do jogo, almeja-se oferecer uma ferramenta pedagógica que seja ao mesmo tempo educativa e engajante.

Além disso, a eficácia da gamificação em promover a literacia financeira, conforme demonstrado por \cite{inchamnan2019gamification}, oferece um respaldo empírico para a abordagem deste projeto. A gamificação não apenas motiva através de recompensas e feedback positivo, mas também tem o potencial de influenciar comportamentos financeiros a longo prazo. Assim, a incorporação de elementos gamificados no jogo didático proposto visa não apenas aumentar o engajamento dos alunos, mas também fornecer um meio eficaz de educação financeira alinhado com a Estratégia Nacional da Educação Financeira (ENEF).


\section{Metodologia}

Para alcançar os objetivos propostos neste trabalho, procederemos à digitalização de um dos três jogos fornecidos para o 5º ano do Ensino Fundamental pela ENEF. Atualmente, esses jogos são apresentados em um formato de livro-jogo, o que pode dificultar a sua intuitividade. A digitalização envolverá a identificação dos cenários e a mapeação dos fluxos de caminhos possíveis. Isso permitirá que os jogadores façam escolhas que os direcionem por diferentes trajetórias, cada uma com suas respectivas consequências.

Antes de iniciar o desenvolvimento do jogo, é essencial elaborar um Game Design Document (GDD). O GDD é um documento estruturado que detalha todos os aspectos do jogo, desde a narrativa, mecânicas, design de níveis, até os recursos gráficos e sonoros. Ele serve como um guia para a equipe de desenvolvimento, garantindo que todos os envolvidos tenham uma visão clara e compartilhada do projeto. A implementação do GDD é crucial para garantir que o jogo seja bem concebido e atenda aos objetivos pedagógicos estabelecidos\cite{arifudin2022gdd}.

O jogo será estruturado em cenários distintos, onde cada um representa uma seção específica da narrativa. Para a implementação, utilizaremos o \textit{framework} RPG MAKER MZ, adequado para a criação de jogos do tipo \textit{Role-Playing Game} (RPG). Esta ferramenta facilita a elaboração de mapas para cada cenário, proporcionando uma experiência imersiva aos jogadores. Uma das principais vantagens deste \textit{framework} é a sua compatibilidade com a plataforma \textit{web}. Isso significa que o jogo pode ser acessado e jogado sem a necessidade de instalação, em qualquer dispositivo com acesso à internet, ampliando assim o seu alcance.

O jogo será disponibilizado em escolas selecionadas de Ensino Fundamental. Após a experiência de jogo, os alunos serão convidados a responder a um questionário de avaliação. A eficácia do jogo será avaliada utilizando a metodologia MEEGA+KIDS\cite{GresseVonWangenheim2020}, que é derivada da metodologia MEEGA+\cite{petri2019meega+}. Esta abordagem oferece um modelo de avaliação específico para jogos educativos, propondo um questionário que mede atributos essenciais relacionados à usabilidade, experiência do usuário e aprendizado.