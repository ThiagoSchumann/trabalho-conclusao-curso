\chapter{Desenvolvimento}

Este capítulo apresenta a visão geral do jogo proposto, seus requisitos e principais elementos, como personagens e ambientes. Também são detalhados os objetivos e mecânicas do jogo e suas cenas.

\section{Visão Geral}

O jogo é chamado ``Mistério Financeiro: A Jornada de Chico''. No início o protagonista Chico se depara com o mistério das despesas crescentes e do desaparecimento de moedas na papelaria de sua família. Motivado por um senso de responsabilidade, ele decide investigar o caso. A investigação começa pelo porão da loja, um local pouco visitado pela sua família.

A narrativa avança através de uma série de desafios e escolhas que Chico deve enfrentar. Essas decisões vão desde escolhas cotidianas, como a compra de um relógio, até interações complexas com personagens, como seu amigo Vicente. Cada escolha tem um impacto na história e nas lições de economia e gestão financeira apresentadas.

À medida que Chico investiga, ele descobre várias causas para os problemas financeiros da papelaria, desde descuidos simples até ações mal-intencionadas de terceiros. O jogo ressalta a importância da atenção aos detalhes e do controle de gastos para a saúde financeira de um negócio.

O jogo é estruturado de forma interativa, oferecendo múltiplos finais baseados nas decisões do jogador ao longo de sua experiência. Isso enfatiza a relevância de escolhas responsáveis e informadas, tanto no jogo quanto na vida real. Neste sentido, o jogo incorpora um forte elemento educacional, enfatizando conceitos de gestão financeira. Ele ensina sobre a importância do planejamento financeiro, economia e investimentos, incentivando os jogadores a refletirem sobre suas próprias decisões financeiras.

\section{Requisitos de Software}

\subsection*{Requisitos Funcionais}
\begin{itemize}
	\item RF1. O jogo deve oferecer uma interface gráfica interativa, simbolizando os diferentes ambientes da história de Chico.
	\item RF2. Incluir um sistema de diálogo interativo com NPCs, como Seu Mário, Vicente, Luiza, entre outros.
	\item RF3. Permitir que o jogador faça escolhas que influenciam o desenvolvimento da história e interações com NPCs.
	\item RF4. Permitir a compra e utilização de itens para progredir na narrativa, principalmente se tratando da Lanterna que será usada para verificar o porão.
	\item RF5. Apresentar múltiplos finais com base nas decisões tomadas pelo jogador, influenciadas pelas interações com NPCs.
	\item RF6. Incorporar elementos de educação financeira dentro da narrativa e desafios, utilizando situações da história de Chico como exemplos.
\end{itemize}

\subsection*{Requisitos Não Funcionais}
\begin{itemize}
	\item RNF1. O jogo deve ter uma interface gráfica atrativa e intuitiva, adequada para a faixa etária do 5º ano.
	\item RNF2. O jogo deve ser otimizado para um desempenho fluido, sem atrasos ou erros técnicos.
	\item RNF3. O jogo deve ser compatível com a plataforma Windows.
	\item RNF4. O jogo deve ter uma trilha sonora e efeitos sonoros imersivos, complementando a experiência visual.
\end{itemize}

\subsection*{Regras de Negócio}
\begin{itemize}
	\item RN1. A história deve se adaptar e mudar com base nas escolhas feitas pelo jogador. (RF3, RF5)
	\item RN2. Os desafios e enigmas devem ser integrados na história e contribuir para o aprendizado sobre finanças. (RF4, RF6)
	\item RN3. Os diálogos e interações com NPCs devem oferecer pistas e informações relevantes para a progressão da história. (RF2)
	\item RN4. O jogo deve promover a conscientização sobre gestão financeira e economia de forma lúdica e educativa. (RF6)
\end{itemize}

\section{Elementos do Jogo}

\subsection{Personagens e Ambientes}
Esta seção detalha os personagens não jogáveis (NPCs) e os mapas que serão fundamentais para a narrativa e jogabilidade do jogo. Cada NPC e mapa pode ser acompanhado de uma imagem e uma descrição detalhada para melhor imersão e compreensão do jogador.

%\subsubsection{NPCs}
\begin{itemize}
	\item Seu Mário: Pai de Chico, dedicado dono da papelaria da família. Enfrenta desafios financeiros e tenta manter o negócio próspero.
	      \begin{figure}[ht]
		      \centering
		      \caption{NPC Seu Mário.}
		      \includegraphics[scale=0.8]{Textuais/Pictures/Seu_Mario.png}
		      \fonte{Criado pelo Autor.}\label{fig:npc-seu-mario}
	      \end{figure}
	\item Pai do Vicente: Aparece na história para repreender seu filho por envolvimento em furtos, refletindo preocupação paterna.
	      \begin{figure}[ht]
		      \centering
		      \caption{NPC Pai do Vicente.}
		      \includegraphics[scale=0.8]{Textuais/Pictures/Pai_Vicente.png}
		      \fonte{Criado pelo Autor.}\label{fig:npc-pai-vicente}
	      \end{figure}
	\item Atendente da Loja de Itens: Caracterizado como um comerciante prestativo, interage com Chico durante a compra do seu relógio.
	      \begin{figure}[ht]
		      \centering
		      \caption{NPC Atendente da Loja de Itens.}
		      \includegraphics[scale=0.8]{Textuais/Pictures/Atendente_loja_Itens.png}
		      \fonte{Criado pelo Autor.}\label{fig:npc-atendente-loja-itens}
	      \end{figure}

	      \newpage

	\item Vicente: Colega de escola de Chico, conhecido por seu comportamento provocativo e envolvimento em pequenos furtos.
	      \begin{figure}[ht]
		      \centering
		      \caption{NPC Vicente.}
		      \includegraphics[scale=0.8]{Textuais/Pictures/Vicente.png}
		      \fonte{Criado pelo Autor.}\label{fig:npc-vicente}
	      \end{figure}
	\item Luiza: Melhor amiga de Chico, sempre oferecendo apoio e aconselhamento nas aventuras e desafios enfrentados por ele.
	      \begin{figure}[ht]
		      \centering
		      \caption{NPC Luiza.}
		      \includegraphics[scale=0.8]{Textuais/Pictures/Luísa.png}
		      \fonte{Criado pelo Autor.}\label{fig:npc-luiza}
	      \end{figure}
	\item Unnamed 1, 2, 3, 4: Personagens secundários, aparecem apenas para ilustrar algumas cenas como o encontro com o pessoal da escola na lanchonete.
	\item Maria José: Mãe de Chico, auxilia na papelaria e compartilha das preocupações financeiras da família.
	      \begin{figure}[ht]
		      \centering
		      \caption{NPC Maria José.}
		      \includegraphics[scale=0.8]{Textuais/Pictures/Maria.png}
		      \fonte{Criado pelo Autor.}\label{fig:npc-maria-jose}
	      \end{figure}

	      \newpage

	\item Josimar: Funcionário da papelaria, inicialmente suspeito de furto, revelando-se inocente e um aliado importante.
	      \begin{figure}[ht]
		      \centering
		      \caption{NPC Josimar.}
		      \includegraphics[scale=0.8]{Textuais/Pictures/Josimar.png}
		      \fonte{Criado pelo Autor.}\label{fig:npc-josimar}
	      \end{figure}
	\item Ratazana: Uma ameaça inesperada no porão da papelaria, adicionando suspense e desafios físicos à narrativa.
	      \begin{figure}[ht]
		      \centering
		      \caption{NPC Ratazana.}
		      \includegraphics[scale=1.8]{Textuais/Pictures/Ratazana.png}
		      \fonte{Hellpug/Emily Lampson.}\label{fig:npc-ratazana}
	      \end{figure}
	\item Atendente do Hospital: Representa um ponto de contato no cenário hospitalar, caso Chico necessite de cuidados médicos.
	      \begin{figure}[ht]
		      \centering
		      \caption{NPC Atendente do Hospital.}
		      \includegraphics[scale=0.8]{Textuais/Pictures/Atendente_Hospital.png}
		      \fonte{\textit{Asset} padrão do RPG Maker MZ.}\label{fig:npc-atendente-hospital}
	      \end{figure}

	      \newpage

	\item Médico: Figura de cuidado e autoridade, que atende Chico no hospital após eventuais incidentes.
	      \begin{figure}[ht]
		      \centering
		      \caption{NPC Médico.}
		      \includegraphics[scale=0.8]{Textuais/Pictures/Medico.png}
		      \fonte{\textit{Asset} padrão do RPG Maker MZ.}\label{fig:npc-medico}
	      \end{figure}
	\item Atendente da Loja de Ferragens: Ajuda Chico na aquisição da lanterna necessária para suas investigações.
	      \begin{figure}[ht]
		      \centering
		      \caption{NPC Atendente da Loja de Ferragens.}
		      \includegraphics[scale=0.8]{Textuais/Pictures/Atendente_Loja_Ferragens.png}
		      \fonte{\textit{Asset} padrão do RPG Maker MZ.}\label{fig:npc-atendente-loja-ferragens}
	      \end{figure}
	\item Cida: Irmã mais nova de Chico, envolvida nos dilemas familiares e curiosa sobre os mistérios da papelaria.
	      \begin{figure}[ht]
		      \centering
		      \caption{NPC Cida.}
		      \includegraphics[scale=0.8]{Textuais/Pictures/Cida.png}
		      \fonte{Criado pelo autor.}\label{fig:npc-cida}
	      \end{figure}
\end{itemize}

\newpage

%\subsubsection{Mapas}
\begin{itemize}
	\item Mundo do Chico: O cenário geral da aventura, incluindo a vizinhança, ruas, e locais frequentados por Chico e seus amigos.

	      \begin{figure}[ht]
		      \centering
		      \caption{Mapa Mundo Chico.}
		      \includegraphics[scale=0.3]{Textuais/Pictures/Mundo_chico.png}
		      \fonte{Criado pelo autor.}\label{fig:mundo-chico}
	      \end{figure}


	\item Loja de Itens: Estabelecimento onde Chico adquire itens como o relógio.

	      \begin{figure}[ht]
		      \centering
		      \caption{Mapa Loja de Itens.}
		      \includegraphics[scale=0.4]{Textuais/Pictures/Loja_itens.png}
		      \fonte{Criado pelo autor.}\label{fig:loja-itens}
	      \end{figure}

	      \newpage

	\item Lanchonete: Espaço social para Chico e seus amigos, propício para conversas e desenvolvimento de subtramas.

	      \begin{figure}[ht]
		      \centering
		      \caption{Mapa Lanchonete.}
		      \includegraphics[scale=0.4]{Textuais/Pictures/Lanchonete.png}
		      \fonte{Criado pelo autor.}\label{fig:lanchonete}
	      \end{figure}

	\item Papelaria da Família: Coração da trama, onde muitos dos mistérios e desafios se concentram.

	      \begin{figure}[ht]
		      \centering
		      \caption{Mapa Papelaria.}
		      \includegraphics[scale=0.5]{Textuais/Pictures/Papelaria_Familia.png}
		      \fonte{Criado pelo autor.}\label{fig:papelaria-familia}
	      \end{figure}

	\item Cozinha da Papelaria: Local onde Chico encontra o Josimar e faz descobertas, como o problema com a geladeira.

	      \begin{figure}[ht]
		      \centering
		      \caption{Mapa da cozinha da papelaria.}
		      \includegraphics[scale=0.5]{Textuais/Pictures/Cozinha_papelaria.png}
		      \fonte{Criado pelo autor.}\label{fig:lanchonete}
	      \end{figure}

	\item Porão da Papelaria: Ambiente sombrio e cheio de mistérios, onde Chico enfrenta a Ratazana.

	      \begin{figure}[ht]
		      \centering
		      \caption{Mapa do porão da papelaria.}
		      \includegraphics[scale=0.4]{Textuais/Pictures/Porao_papelaria.png}
		      \fonte{Criado pelo autor.}\label{fig:porao-papelaria}
	      \end{figure}

	      \newpage

	\item Hospital: Local de cuidado e recuperação, podendo ser cenário de eventos dramáticos, como o ataque da ratazana.

	      \begin{figure}[ht]
		      \centering
		      \caption{Mapa do hospital.}
		      \includegraphics[scale=0.5]{Textuais/Pictures/Hospital.png}
		      \fonte{Criado pelo autor.}\label{fig:hospital}
	      \end{figure}

	      \newpage

	\item Loja de Ferragens: Fornecedora da lanterna para as investigações de Chico.

	      \begin{figure}[ht]
		      \centering
		      \caption{Mapa da loja de ferragens.}
		      \includegraphics[scale=0.3]{Textuais/Pictures/Loja_Ferragens.png}
		      \fonte{Criado pelo autor.}\label{fig:loja-ferragens}
	      \end{figure}

	\item Casa do Chico: Espaço de união e diálogo da família, proporcionando um contraste com as áreas de mistério e aventura.

	      \begin{figure}[ht]
		      \centering
		      \caption{Mapa da casa do Chico.}
		      \includegraphics[scale=0.3]{Textuais/Pictures/Casa_chico.png}
		      \fonte{Criado pelo autor.}\label{fig:casa-chico}
	      \end{figure}
\end{itemize}
