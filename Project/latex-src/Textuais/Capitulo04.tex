\chapter{Considerações finais}

Este trabalho buscou abordar a relevância da educação financeira para o público infantil, um tema de grande importância na sociedade contemporânea. Foi desenvolvido um jogo digital educativo intitulado \textit{Mistério Financeiro: A Jornada de Chico}, que combina aspectos lúdicos com o ensino de conceitos financeiros básicos. Este jogo foi estruturado com base na narrativa da Estratégia Nacional de Educação Financeira (ENEF), visando proporcionar uma experiência educativa rica e engajadora para crianças do 5º ano do Ensino Fundamental.

O jogo apresenta uma estrutura de RPG interativo, onde as escolhas do jogador influenciam diretamente o andamento da história. O enredo é centrado em Chico, um menino que se depara com desafios financeiros em sua família e decide investigar as causas. As decisões tomadas pelo jogador ao longo da narrativa não apenas afetam o resultado da história, mas também oferecem oportunidades de aprendizado sobre gestão de recursos, economia e planejamento financeiro.

A experiência de desenvolvimento deste projeto permitiu a integração de práticas pedagógicas com elementos lúdicos, resultando em um recurso didático que pode ser particularmente eficaz no ensino de conceitos financeiros para crianças. Essa abordagem é consistente com a literatura atual, que enfatiza a importância da \textit{gamificação} na educação e a necessidade de métodos inovadores que engajem os estudantes e complementem os currículos tradicionais.

\section{Desenvolvimentos Futuros}
Embora o jogo \textit{Mistério Financeiro: A Jornada de Chico} represente um passo significativo na direção de uma educação financeira mais engajadora para crianças, há dois desenvolvimentos futuros essenciais que podem expandir e enriquecer ainda mais este projeto:

\subsection*{Validação com Estudantes}
A aplicação prática do jogo em um ambiente escolar, com a participação direta de estudantes do Ensino Fundamental, é importante para avaliar sua eficácia educacional. Testes em sala de aula permitirão observar a interação dos alunos com o jogo, avaliar o grau de engajamento, o entendimento dos conceitos financeiros apresentados e a capacidade de aplicar o conhecimento adquirido em situações práticas. Além disso, o retorno dos professores e estudantes serão valiosos para identificar dificuldades, refinar o jogo e garantir que o conteúdo esteja alinhado com os objetivos pedagógicos.

\subsection*{Implementação de Outras Histórias da ENEF}
A expansão do jogo com a inclusão de outras narrativas e cenários presentes nos materiais da ENEF enriquecerá o espectro de experiências e aprendizados oferecidos. Essa diversificação permitirá abordar uma gama mais ampla de tópicos financeiros e proporcionar diferentes contextos e desafios aos jogadores. Além disso, oferecerá aos alunos a oportunidade de explorar diversas situações financeiras, cada uma com seus próprios desafios e lições, aumentando a abrangência e a profundidade do aprendizado.

\section{Reflexão Final}
O desenvolvimento de "Mistério Financeiro: A Jornada de Chico" representa um esforço para tornar a educação financeira mais acessível e atraente para o público infantil. O uso de jogos educativos digitais como ferramentas pedagógicas é uma abordagem promissora que pode complementar os métodos de ensino tradicionais, proporcionando um ambiente de aprendizagem mais dinâmico e interativo. A implementação deste projeto e os desenvolvimentos futuros propostos têm o potencial de contribuir significativamente para a formação de jovens mais conscientes e preparados para gerir suas finanças pessoais de forma responsável.