% ---
% RESUMOS
% ---

% resumo em português
\setlength{\absparsep}{18pt} % ajusta o espaçamento dos parágrafos do resumo
\begin{resumo}
	Este trabalho explora a implementação da educação financeira para crianças através do desenvolvimento de um jogo digital educativo, "Mistério Financeiro: A Jornada de Chico", utilizando a tecnologia RPG Maker MZ. Direcionado para alunos do 5º ano do Ensino Fundamental, o jogo está alinhado à Estratégia Nacional de Educação Financeira (ENEF), promovendo o ensino de conceitos financeiros por meio de uma narrativa interativa de RPG. As escolhas dos jogadores influenciam o enredo, proporcionando uma experiência educativa imersiva e reflexiva sobre gestão de recursos, economia e planejamento financeiro. A análise crítica revela a eficácia do jogo em engajar o público-alvo e destaca a importância de futuros testes práticos e da futura expansão do conteúdo narrativo. O estudo sublinha a relevância de métodos de ensino inovadores e interativos na formação financeira de crianças, visando a construção de uma sociedade mais consciente e equilibrada economicamente.

	\textbf{Palavras-chave}: Educação financeira. Jogos educativos digitais. Ensino Fundamental. RPG Maker MZ. Estratégia Nacional de Educação Financeira.

\end{resumo}