\chapter{Introdução} 
A educação financeira é um pilar crucial na formação integral do indivíduo, proporcionando uma vida mais equilibrada e tranquila à medida que este desenvolve compreensão profunda e habilidades práticas para gerenciar suas finanças no cotidiano. Em um mundo onde a economia é notoriamente volátil, o conhecimento financeiro é uma ferramenta indispensável para navegar por cenários de crise e recessão, conforme delineado pela \citeshort{OECD}.
Crianças e adolescentes, sendo altamente influenciáveis, são frequentemente o alvo de campanhas de marketing intensivas que promovem o consumismo. Essas estratégias de marketing, muitas vezes, não apenas criam hábitos de consumo, mas também confundem as noções de necessidade e desejo nos jovens, como observado por \citeshort{Junger_Medeiros_Moura_Barrocal_Amaral_2019}. Esta confusão pode predispor as crianças a se tornarem adultos inclinados ao consumismo impulsivo, sublinhando a necessidade de uma educação financeira robusta desde tenra idade.
Reconhecendo a importância da educação financeira, o Governo Federal brasileiro implementou a Estratégia Nacional da Educação Financeira (ENEF) e o Fórum Brasileiro de Educação Financeira (FBEF) através dos Decretos 7.397/2010 e 10.393/2020, respectivamente \cite{decreto_10939}. Essas iniciativas visam integrar a educação financeira ao currículo escolar desde os primeiros anos, fornecendo aos estudantes ferramentas essenciais para uma gestão financeira responsável e consciente.
A implementação da educação financeira nas escolas não beneficia apenas os estudantes, mas também exerce um impacto positivo significativo sobre os pais, especialmente aqueles em famílias desfavorecidas \cite{Frisancho2023Spillover}. Além disso, a introdução de tarefas matemáticas alinhadas com conceitos financeiros desde os primeiros anos escolares podem estabelecer uma fundação sólida para o desenvolvimento da literacia financeira entre as crianças \cite{Ozkale2023Designing}.
Para consolidar a importância da educação financeira desde a infância, é imperativo considerar que os hábitos e percepções formados durante este período são cruciais para o comportamento financeiro na vida adulta \cite{AbuBakarAh2016Importance}. Uma educação financeira estruturada e integrada ao currículo escolar não apenas empodera os estudantes com conhecimento valioso, mas também promove uma sociedade mais consciente e responsável financeiramente, preparada para tomar decisões informadas e estratégicas em relação às suas finanças.

\section{Problema}
Embora o ensino tradicional tenha se mostrado eficaz em capturar a atenção dos alunos no passado, observa-se uma tendência decrescente em sua eficácia, especialmente no que tange à aprendizagem. Aulas expositivas, por vezes, não conseguem engajar os alunos devido à natureza variada dos conteúdos, resultando em desinteresse por parte dos estudantes. Nesse contexto, a inovação educacional emerge como um campo promissor para o desenvolvimento de estratégias que tornem o processo de aprendizagem mais atraente.
Conforme dados de \citeshort{SERASA_Setembro_2022}, em setembro de 2022, aproximadamente 69 milhões de brasileiros estavam inadimplentes, o que corresponde a mais de 30\% da população, segundo estimativas do \cite{IBGE_populacao}. Esses números sublinham a urgência de uma educação financeira robusta desde a infância, visando uma redução futura na taxa de inadimplência no país.
O emprego de jogos digitais como ferramentas educacionais oferece uma experiência de aprendizagem mais envolvente e prazerosa para os alunos. Os jogos não apenas capturam o interesse dos estudantes, mas também fomentam a criatividade, a imaginação e promovem a socialização, uma vez que incentivam o trabalho em equipe para atingir objetivos comuns. Contudo, é crucial que o uso de jogos digitais na educação seja moderado e estratégico, necessitando de uma avaliação prévia por parte dos educadores para assegurar sua adequação à faixa etária dos alunos \cite{Cruz_Araujo_Andrye_Galvao_Madeira_2022}.
Diante do exposto, identifica-se a necessidade de implementar um jogo digital voltado para o ensino de conceitos de educação financeira para alunos do Ensino Fundamental, como meio de auxiliar no processo de aprendizagem desses importantes conteúdos.

\section{Objetivos}
Esta seção apresenta o objetivo geral e os objetivos específicos do presente trabalho.

\subsection{Objetivo Geral}
implementar um jogo sério, destinado a facilitar o processo de aprendizado de conceitos fundamentais de educação financeira nas escolas. 

\subsection{Objetivos Específicos}
\begin{enumerate}[noitemsep,nosep,labelindent=\parindent,leftmargin=*,label={\alph*}) ]
	\item Analisar e compreender o material fornecido pela Estratégia Nacional da Educação Financeira (ENEF) destinado ao 5º ano do Ensino Fundamental \cite{Educacao_financeira_nas_escolas}.
	\item Projetar e implementar um jogo digital educativo, baseando-se na primeira história contida no material previamente estudado.
	\item Implementar o uso do jogo desenvolvido em turmas do 5º ano do Ensino Fundamental.
	\item Executar o teste proposto pelo modelo MEEGA+KIDS \cite{GresseVonWangenheim2020} nas turmas que utilizaram o jogo.
	\item Avaliar a eficácia do jogo desenvolvido em termos de transmissão de conceitos financeiros, utilizando como referência os critérios estabelecidos pela metodologia MEEGA+KIDS\cite{GresseVonWangenheim2020}.
\end{enumerate}
          

\section{Justificativa}
As práticas lúdicas são atividades que trazem aprendizado através de brincadeiras ou jogos. Elas são importantes na formação do indivíduo, pois fomentam aptidões tais como a criatividade, empatia e interação social. Isso é mais efetivo no aprendizado da criança do que as práticas convencionais, pois é mais simples manter uma criança atenta durante uma brincadeira em comparação a uma explicação mais séria sobre conhecimentos similares, principalmente considerando a disponibilidade de outras opções de entretenimento disponíveis \cite{Santos_Thayna_da_silva_2021}.
Visando tomar proveito dos benefícios das atividades lúdicas, será realizada a implementação da versão digital do jogo disponibilizado no material do 5º ano do Ensino Fundamental para ser mais lúdico no sentido de permitir ao aluno interagir de maneira mais imersiva em relação ao modelo convencional. Atualmente, o referido jogo precisa ser jogado folhando as páginas do livro com muito texto e poucas imagens, o que de certa forma diminui o interesse da criança pela atividade.

\section{Metodologia}
A fim de atender aos objetivos do presente trabalho, será realizada a digitalização de um de três jogos presentes no material disponibilizado para o 5º ano do Ensino Fundamental pela ENEF. No momento, esses jogos estão disponíveis em um formato de livro-jogo, que o torna menos intuitivo. A digitalização do jogo consistirá no levantamento dos cenários e fluxos de caminhos possíveis, permitindo ao jogador tomar decisões que o levarão por caminhos diferentes e implicarão em consequências em seu andamento. O jogo será divido em cenários, onde cada cenário representa uma pequena parte da história.
A implementação do jogo ocorrerá utilizando o \textit{framework} RPGJS, que permite a criação de um \textit{Role-Playing Game} (RPG), isso viabiliza a criação de mapas com cada cenário, provendo uma experiência imersiva para o usuário. Outro ponto importante na escolha da tecnologia foi a plataforma de destino, sendo a \textit{web}, que capacita este \textit{framework} para disponibilizar o jogo sem a necessidade da instalação e em qualquer dispositivo que tenha acesso à internet, o que permite atingir um público maior.
A disponibilização do jogo será realizada em algumas escolas de Ensino Fundamental a fim de permitir que os alunos joguem o jogo e após isso estejam aptos a responder o questionário de avaliação.
A avaliação da eficiência do jogo ocorrerá através da metodologia MEEGA+KIDS\cite{GresseVonWangenheim2020}, que tem como base a metodologia MEEGA+\cite{petri2019meega+}. Ela se resume a um modelo de avaliação de jogos voltados ao ensino, de maneira que propõe um questionário baseado em conceitos consolidados de mensuração de atributos mínimos de usabilidade, experiência e aprendizado.